\documentclass[11pt, a4paper]{article}

% Packages:
\usepackage[
    ignoreheadfoot, % set margins without considering header and footer
    top=2 cm, % seperation between body and page edge from the top
    bottom=2 cm, % seperation between body and page edge from the bottom
    left=1.1 cm, % seperation between body and page edge from the left
    right=1.5 cm, % seperation between body and page edge from the right
    footskip=1.0 cm, % seperation between body and footer
    % showframe % for debugging 
]{geometry} % for adjusting page geometry
\usepackage[explicit]{titlesec} % for customizing section titles
\usepackage{tabularx} % for making tables with fixed width columns
\usepackage{array} % tabularx requires this
\usepackage[dvipsnames]{xcolor} % for coloring text
\definecolor{primaryColor}{RGB}{0, 64, 144} % define primary color
\usepackage{enumitem} % for customizing lists
\usepackage{fontawesome5} % for using icons
\usepackage{amsmath} % for math
\usepackage[
    pdftitle={Matěj Volf's CV},
    pdfauthor={Matěj Volf},
    pdfcreator={LaTeX},
    colorlinks=true,
    urlcolor=primaryColor
]{hyperref} % for links, metadata and bookmarks
\usepackage[pscoord]{eso-pic} % for floating text on the page
\usepackage{calc} % for calculating lengths
\usepackage{bookmark} % for bookmarks
\usepackage{lastpage} % for getting the total number of pages
\usepackage{changepage} % for one column entries (adjustwidth environment)
\usepackage{paracol} % for two and three column entries
\usepackage{ifthen} % for conditional statements
\usepackage{needspace} % for avoiding page brake right after the section title
\usepackage{iftex} % check if engine is pdflatex, xetex or luatex
% Ensure that generate pdf is machine readable/ATS parsable:
\ifPDFTeX
    \input{glyphtounicode}
    \pdfgentounicode=1
    \usepackage[T1]{fontenc}
    \usepackage[utf8]{inputenc}
    \usepackage{lmodern}
\fi
\usepackage{graphicx} % for the profile picture

% Some settings:
\AtBeginEnvironment{adjustwidth}{\partopsep0pt} % remove space before adjustwidth environment
\pagestyle{empty} % no header or footer
\setcounter{secnumdepth}{0} % no section numbering
\setlength{\parindent}{0pt} % no indentation
\setlength{\topskip}{0pt} % no top skip
\setlength{\columnsep}{0.15cm} % set column seperation
\pagenumbering{gobble} % no page numbering

\titleformat{\section}{
    % avoid page braking right after the section title
    \needspace{4\baselineskip}
    % make the font size of the section title large and color it with the primary color
    \Large\color{primaryColor}
}{
}{
}{
    % print bold title, give 0.15 cm space and draw a line of 0.8 pt thickness
    % from the end of the title to the end of the body
    \textbf{#1}\hspace{0.15cm}\titlerule[0.8pt]\hspace{-0.1cm}
}[] % section title formatting

\titlespacing{\section}{
    % left space:
    -1pt
}{
    % top space:
    0.4 cm
}{
    % bottom space:
    0.2 cm
} % section title spacing

% \renewcommand\labelitemi{$\vcenter{\hbox{\small$\bullet$}}$} % custom bullet points
\newenvironment{highlights}{
    \begin{itemize}[
        topsep=0.08 cm,
        parsep=0.05 cm,
        partopsep=0pt,
        itemsep=0pt,
        leftmargin=0.5 cm
    ]
}{
    \end{itemize}
} % new environment for highlights

\newenvironment{header}{
    \setlength{\topsep}{0pt}\par\kern\topsep\centering\color{primaryColor}\linespread{1.5}
}{
    \par\kern\topsep
} % new environment for the header

\newcommand{\placelastupdatedtext}{% \placetextbox{<horizontal pos>}{<vertical pos>}{<stuff>}
  \AddToShipoutPictureFG*{% Add <stuff> to current page foreground
    \put(
        \LenToUnit{\paperwidth-1.5 cm-0 cm+0.05cm},
        \LenToUnit{\paperheight-1.0 cm}
    ){\vtop{{\null}\makebox[0pt][c]{
        \small\color{gray}\textit{Naposledy aktualizováno: leden 2025}\hspace{\widthof{Naposledy aktualizováno: leden 2025}}
    }}}%
  }%
}%

% save the original href command in a new command:
\let\hrefWithoutArrow\href

% new command for external links:
\renewcommand{\href}[2]{\hrefWithoutArrow{#1}{\ifthenelse{\equal{#2}{}}{ }{#2 }\raisebox{.15ex}{\footnotesize \faExternalLink*}}}


\begin{document}
    \newcommand{\AND}{\unskip
        \cleaders\copy\ANDbox\hskip\wd\ANDbox
        \ignorespaces
    }
    \newsavebox\ANDbox
    \sbox\ANDbox{}

    \placelastupdatedtext

    \begin{header}
        \fontsize{20 pt}{20 pt}
        \textbf{Matěj Volf}

        \vspace{0.2 cm}

        \normalsize
        \mbox{\hrefWithoutArrow{mailto:matej@mvolfik.com}{{\footnotesize\faEnvelope[regular]}\hspace*{0.13cm}matej@mvolfik.com}}%
        \kern 0.25 cm%
        \AND%
        \kern 0.25 cm%
        \mbox{\hrefWithoutArrow{tel:+420-778-259-280}{{\footnotesize\faPhone*}\hspace*{0.13cm}+420 778 259 280}}%
    \end{header}
    \vspace{0.5 cm}

\newenvironment{mysection}[2][]{
    \vspace{0.4 cm}
    \begin{adjustwidth}{
        0 cm + 0.00001 cm
    }{
        0 cm + 0.00001 cm
    }
    \setcolumnwidth{3.6 cm, 0.3 cm, \fill}
    \begin{paracol}{3}
    \raggedleft
    \large\color{primaryColor}\textbf{#2}\normalsize\color{black}
    \switchcolumn
    \switchcolumn
    \raggedright
}{
    \end{paracol}
    \end{adjustwidth}
} % new environment for two column entries

        % \begin{threecolentry}{\textbf{}}{
        %     září 2014 – červen 2022
        % }
        %     \textbf{Gymnázium Pierra de Coubertina Tábor}, \newline Osmileté všeobecné gymnázium
        %     \begin{highlights}
        %         \item Aktivní účast na oborových soutěžích v matematice, zeměpisná olympiáda, lingvistická olympiáda
        %     \end{highlights}
        % \end{threecolentry}

\newcommand{\educationheader}[1]{
    \raggedleft
    #1

    \vspace{-0.48 cm}
    \raggedright
}

    \begin{mysection}{Vzdělání}
        \educationheader{od října 2024}
        \textbf{Filozofická fakulta Univerzity Karlovy}

        \textbf{Bc.}, obor Germánská a severoevropská studia
        \begin{itemize}[
            topsep=0pt,
            parsep=0pt,
            partopsep=0pt,
            itemsep=0pt,
            % no bullet
            label={}
        ]
            \item Skandinavistika se zaměřením na dánštinu
            \item Finština
        \end{itemize}

        \begin{highlights}
            \item výuka obou jazyků rodilými mluvčími
            \item vzdělání v literatuře, kultuře a historii obou severských regionů
        \end{highlights}

        \vspace{0.4 cm}

        \educationheader{2022 -- 2025}
        \textbf{Matematicko-fyzikální fakulta Univerzity Karlovy}

        \textbf{Bc.}, obor Informatika

        \begin{highlights}
            \item očekávané dokončení v červnu 2025
        \end{highlights}

        \vspace{0.4 cm}

        \educationheader{2014 -- 2022}
        \textbf{Gymnázium Pierra de Coubertina v Táboře}

        Všeobecné osmileté gymnázium

        \begin{highlights}
            \item Aktivní účast na oborových soutěžích v matematice, zeměpisné olympiádě a lingvistické olympiádě
        \end{highlights}
    \end{mysection}

    \begin{mysection}{Dobrovolnictví}
        \educationheader{2018 -- 2021}
        \textbf{\href{https://eyp.org}{European Youth Parliament}}

        \begin{highlights}
            \item Účast na celkem 8 studentských konferencích v Česku i zahraničí
            \begin{itemize}[leftmargin=7pt, topsep=0pt]
                \item debata v komisích nad evropskými tématy, představení závěrů na valném shromáždění, pohotové zodpovídání dotazů a kritiky
                \item celý program probíhá v anglickém jazyce
            \end{itemize}
            \vspace{4pt}
            \item Vedoucí komise na 2. Digitální národní konferenci EPM v ČR
            \begin{itemize}[leftmargin=7pt, topsep=0pt]
                \item vypracování akademické přípravy a úvodu do tématu elektronických voleb
                \item facilitace skupiny 12 delegátů ke konsenzu na rezoluci
            \end{itemize}
            \vspace{4pt}
            \item Hlavní organizátor 10. Regionální výběrové konference EPM v ČR -- Třebíč 2021
            \begin{itemize}[leftmargin=7pt, topsep=0pt]
                \item kompletní organizace čtyřdenní hybridní (prezenční a online) konference pro 90~účastníků
                \item vedoucí týmu 8 organizátorů
            \end{itemize}
            \vspace{4pt}
            \item Rozvoj schopností práce v týmu, leadershipu, veřejného vystupování a projektového řízení v mezinárodním, anglicky mluvícím prostředí
        \end{highlights}

        \vspace{0.4 cm}

        \educationheader{od března 2023}
        \textbf{Programátorský tábor \href{https://protab.cz}{Protab}}

        Lektor a organizátor letního tábora pro středoškoláky
    \end{mysection}

    \begin{mysection}{Jazykové schopnosti}
            \vspace{0.45 cm}
            \textbf{angličtina} (C2 certifikát), \textbf{němčina} (B1)
    \end{mysection}


    \begin{mysection}{Zájmy}
        IT, sport, urbanismus a městská cyklistika, politika a společenské dění
    \end{mysection}

\end{document}